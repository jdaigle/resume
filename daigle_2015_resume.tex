%% start of file `template.tex'.
%% Copyright 2006-2010 Xavier Danaux (xdanaux@gmail.com).
%% Copyright 2010-2011 Mark Liu (markwayneliu@gmail.com).
%% Copyright 2015 Joseph Daigle (joseph.daigle@gmail.com).
%
% This work may be distributed and/or modified under the
% conditions of the LaTeX Project Public License version 1.3c,
% available at http://www.latex-project.org/lppl/.

\documentclass[11pt,letterpaper,sans]{moderncv}

\usepackage{verbatim}

% moderncv themes
\moderncvstyle{classic}
\moderncvcolor{blue}

% character encoding
\usepackage[utf8]{inputenc}                   % replace by the encoding you are using

% adjust the page margins
\usepackage[scale=0.8]{geometry}
%\setlength{\hintscolumnwidth}{3cm}						% if you want to change the width of the column with the dates
%\AtBeginDocument{\setlength{\maketitlenamewidth}{6cm}}  % only for the classic theme, if you want to change the width of your name placeholder (to leave more space for your address details
%\AtBeginDocument{\recomputelengths}                     % required when changes are made to page layout lengths


% personal data
\firstname{Joseph}
\familyname{Daigle}
\email{joseph.daigle@gmail.com}
\phone{+1~(678)~296~6166}
\extrainfo{%
  \homepagesymbol\url{https://stackoverflow.com/users/507/joseph-daigle} \\
  \homepagesymbol\url{https://github.com/jdaigle}
}

% to show numerical labels in the bibliography; only useful if you make citations in your resume
%\makeatletter
%\renewcommand*{\bibliographyitemlabel}{\@biblabel{\arabic{enumiv}}}
%\makeatother

%\nopagenumbers{}                             % uncomment to suppress automatic page numbering for CVs longer than one page
%----------------------------------------------------------------------------------
%            content
%----------------------------------------------------------------------------------
\begin{document}
\maketitle

\section{Experience}

\cventry{2013--Present}{Director of Development}{Clearwave Corporation}{Atlanta, GA}{}{
{\url{http://www.clearwaveinc.com/}}\newline{}\newline{}
Promoted to a position to manage the entire software development team and software development life-cycle. Responsibilities include hiring new developers, mentoring both junior and senior developers, managing the backlog of product requirements and feature requests, managing the release schedule, leading code reviews and sprint planning meetings, and maintaining architectural consistency across all products and applications. Day-to-day activities require keeping developers productive by reducing the overhead of meetings and disruptions, as well as continuing to work side-by-side with developers to quickly implement new features on time.
\begin{itemize}
\item Introduced guidelines for client implementations team to provide requirements and specifications for back-end integrations, such as HL7 and VPN connections.
\item In charge of researching and designing the requirements for the server and network equipment in a brand new data center. Tasked to lead the final migration of applications, database, and network connections between data centers.
\end{itemize}
}

\cventry{2009--2013}{Senior Software Engineer}{Clearwave Corporation}{Atlanta, GA}{}{
{\url{http://www.clearwaveinc.com/}}\newline{}\newline{}
Software development engineer working on patient registration solutions for healthcare providers. Primarily web based applications hosted from a "cloud" data center. Back end solutions include EDI transactions for health insurance eligibility verification, payment processor integration, and bidirectional HL7 integrations for patient registration data flow to and from customer practice management systems.
\newline{}
\begin{itemize}
\item Introduced "agile" methodologies and TDD practices that continue to flourish to this day.
\item Worked on a team to design and implement a new HTML-based Kiosk application to replace the aging Flash-based application. Server side frameworks and utilities include ASP.NET MVC and NHibernate. Client side scripting was handled through JQuery and various plug-ins.
\item Designed a custom "Kiosk Browser" using WinForms and embedded IE Web Control that is installed on deployed kiosks and is used to interact with the web application.
\item Implemented kiosk hardware integrations including magnetic stripe readers (MSR) for payment and a camera for scanning driver's licenses and medical insurance cards.
\item Implemented a software solution to automatically scan and read information from driver's licenses and medical insurance cards through optical character recognition (OCR).
\item Redesigned Provider Portal application to use modern development strategies including ASP.NET MVC and client side scripting libraries such as JQuery.
\item Design and implemented a new EDI engine to integrate with healthcare insurance companies for electronic eligibility verification. The system is designed around NServiceBus utilizing a publish/subscribe architecture to scale to millions of transactions per month.
\end{itemize}
}

\cventry{2008--2009}{Software Engineer}{GeoFields, Inc.}{Atlanta, GA}{}{
{\url{http://www.geofields.com/}}\newline{}\newline{}
Enterprise software engineer for the industry leader in transmission pipeline data management software and services. Worked in multiple small teams that delivered a comprehensive enterprise software solution. Responsibilities included all aspects of the software process from design through development, quality assurance and release. Primarily utilized .NET, C\#, ASP.NET, ArcObjects, ArcDesktop, SQL Server, Oracle, LLBLGen.
\newline{}
\begin{itemize}
\item Lead developer on the product DataFrame Data Maintenance (DFDM). The product is designed as an extension to ESRI’s ArcMap application to maintain linearly-referenced transmission pipeline data in a customized geodatabase. The software consists of a .NET WinForms based front end, written in C\#, with a shared back end library containing the data access layer and domain object layer. Additionally, intense ArcObjects programming and business logic layer design was involved to build the user interface and interact with the data model. This product was demoed at the 2008 GeoFields User Conference.
\item Developer on the product Facility Explorer (FE). An ASP.NET based web application which utilizes ESRI mapping technology to dynamically display transmission pipeline assets, as well as a reporting tool. It relies heavily on custom JavaScript libraries to enable rich user interaction with the map. Also participated in support of a legacy ASP version of the software.
\item Lead in the design and development of a large enterprise foundation code base to be reused by several other software applications. Much of the work included building a dynamic data access layer and business logic layer. Several newer .NET technologies were utilized including WCF and LINQ.
\end{itemize}
}

\cventry{2005-2007}{IT Analyst / IT Support Specialist}{Southern Regional Education Board}{Atlanta, GA}{}{
{\url{http://www.sreb.org/}}\newline{}\newline{}
Worked during alternating semesters through Georgia Tech's Cooperative Education Program. As an IT support specialist, supported 150 users and workstations. Additionally designed and implemented ASP.NET web applications for internal and external client projects.
\newline{}
\begin{itemize}
\item Designed and developed an internal content management system for the web and desktops based on Microsoft SharePoint. This utilized several custom ASP.NET web parts as well as integration with Active Directory for authentication and authorization. An ASP.NET web application was built to enable external user registration within an isolated Active Directory application domain, and email support.
\item Worked on several projects to create classic ASP and ASP.NET web surveys answered by thousands of students and educators across the country. A custom framework was designed to facilitate the implementation and deployment of these surveys for various teams within the organization.
\item Built a 1TB storage cluster with fail-over and data replication using DFS and DFSR to host and share thousands of internal documents and publications across the organization.
\end{itemize}
}

\cventry{2004}{Systems Support Specialist / IT Intern}{Porex Corporation}{Atlanta, GA}{}{
{\url{http://www.porex.com/}}\newline{}\newline{}
Responsibilities included supporting over 250 users and client workstations across work 3 work sites and systems administration of several Active Directory and Exchange servers.
\newline{}
\begin{itemize}
\item Designed and implemented a large scale user desktop imaging/deployment system utilizing Windows deployment technologies including sysprep and 3rd part tools such as Symantec Ghost.
\item Developer for a Java based network monitoring system. Utilized several open source products including OpenNMS and other assets management tools, customized for the environment and monitoring requirements.
\end{itemize}
}

\section{Education}
\cventry{2003--2007}{BS, Computer Science}{Georgia Institute of Technology}{Atlanta, GA}{}{}
\cvitem{}{Specialized in Networks, Communication, and Systems.}

\section{Technical Experience}
\subsection{Extremely Proficient With}
\cvline{languages}{C\#, XML/XLST, T-SQL, JavaScript, HTML, CSS, JSON}
\cvline{technologies}{GIT, Subversion, TFS, ASP.NET, Windows Server, IIS, MS SQL Server, Linux/CentOS, JQuery, NHibernate, Active Directory, MS Exchange Server, HAProxy}
\subsection{Have Experience With}
\cvline{languages}{Java, VB.NET, C++, PHP, Ruby, Scheme, Python, Perl, F\#}
\cvline{technologies}{OSX, Apache, Mercurial, ESRI, Lucene, Solr, ElasticSearch, Redis, Cassandra, RavenDB}

\end{document}
